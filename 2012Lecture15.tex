%!TEX root = InfoSec.tex
% Lecture 15: 12 November 2012
\sektion{15}{Network security: protecting the infrastructure}
\subsektion{Physical infrastructure}
destruction by backhoes and rodents, problem
of getting information off of physical wires, physical security of places where
switching/routing is done
\subsektion{Routing}
\begin{tabular}{|c|}
\hline
sender IP\\
\hline
dest IP\\
\hline
 \\
IP packet\\
 \\
\hline
\end{tabular}
\begin{itemize}
\item hop-by-hop
\item adaptive: routers talk to each other in real-time to figure out which
    routes are good to get to particular destinations (router gossips with
            neighbors to find good routes)
    \begin{itemize}
    \item bad gossip leads to bad routes - malicious/censorship, misconfiguration
    \end{itemize}
\item addressing some of the problems, but still hard to stop mistakes or malice
    \begin{itemize}
    \item prevent spoofed messages by authenticating - doesn't address genuine
        but wrong information
    \item check consistency
    \item hard-wire certain information - also helps with maintaining business
        deals
    \end{itemize}
\end{itemize}
\subsektion{DNS}
\texttt{www.princeton.edu} $\rightarrow$ 128.112.1.17

Lookup uses UDP-based protocol:
\begin{itemize}
    \item request packet $\longrightarrow$
    \item reply packet $\longleftarrow$
\end{itemize}
Problem: spoofed reply - attacker sends (forged) reply packet\\
Defense (in place): request contains 16-bit query ID (QID), reply contains QID
\begin{itemize}
    \item Old days: generated QIDs sequentially - attacker can learn QID then
        predict for the near future
    \item Fix: generate QIDs pseudorandomly
    \item Kaminksy attack: attacker gets $2^{16}$ guesses within a few seconds
        -- big problem
    \item Fix: DNSSEC (existed, just not deployed)
    \begin{itemize}
        \item Idea: reply is digitally signed, reply includes not only serverID
            of authoritative server for the name, but also that server's public
            key
        \item To look up \texttt{www.princeton.edu}:
        \begin{itemize}
            \item ask root server: lookup(www.princeton.edu)
            \item reply: (.edu, IPaddr$_1$, public key for .edu) signed by root
                (hardcoded public key)
            \item ask IPaddr$_1$: lookup(www.princeton.edu)
            \item reply: (.princeton.edu, IPaddr$_2$, public key for
                .princeton.edu) signed by .edu (use public key received from
                root)
        \end{itemize}
    \end{itemize}
\end{itemize}
\subsektion{TCP}
Provides reliable connection between machines; built on top of IP

Vulnerabilities:
\begin{itemize}
    \item address spoofing - false source IP address
    \item solution: reply to alleged address, put in something hard for attacker
        to guess
    \item tampering + eavesdropping - MITM along the route
    \item solution: encrypt traffic
    \item jamming - sending a forged error packet, flooding with traffic (TCP
        SYN packets, result in half-open connections but can't open real
        connections since fill table -- send traffic that's costly to process)
    \item solution: filter these packets, don't believe first reset packet sent,
        be skeptical of these kinds of packets, \dots
\end{itemize}
Checking packets for sender/receiver plausibility: routers throw away packets
with addresses that seem to be wrong (what addresses should be incoming/outgoing
traffic)
\subsektion{SSL/TLS}
Secure Sockets Layer, Transport Layer Security

Provides secure (confidential and integrity-checked) stream on top of TCP and
authentication of the server (usually) and the client (rarely)
\begin{itemize}
    \item server is usually a well-known entity
    \item client is usually just some random person who doesn't need to be
        authenticated, or use username/password to authenticate
\end{itemize}
HTTPS = HTTP on top of SSL/TLS

Authentication of the server:
\begin{itemize}
    \item server signs transcript of DH handshake
    \item how does client know server's public key? certificate authority
    \begin{itemize}
        \item problem with trusting certificate authority and compromises of the
            certificate authorities
        \item certificate authority needs to make sure that legitimate owner is
            asking to get the certificate (need due diligence)
        \item certificate authority can use root key to delegate power to
            another key used to sign certificates and can sell this power
            (PROBLEM! - no public accountability for these delegations)
    \end{itemize}
\end{itemize}
\subsektion{IPSEC}
Attempt to build integrity and confidentiality in at level of IP packets.
It uses MAC and ESP (encrypted security payload), but is not widely emplyed
today. Instead most new ideas address fixes at the SSL/TLS level.
