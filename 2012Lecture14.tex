%!TEX root = InfoSec.tex
% Lecture 14: 07 November 2012
\sektion{14}{Trusted Computing: Platform support for security}
Addresses ``attack from below''

Tamper-resistant devices (not ``\underline{tamper-proof}'', since that doesn't
        exist)

Examples:
\begin{itemize}
    \item smart card
        \begin{itemize}
            \item costs \$1
            \item 10K-100'sK RAM
            \item 8- or 16- bit CPU
            \item destructive attack to get crypto keys by sanding layers off to
                    reconstruct 3D model of circuitry
        \end{itemize}
    \item TR (tamper resistant) ``brick'' device
        \begin{itemize}
            \item internal power
            \item active defense against physical attack (cutting, poking,
                    probing, \dots)
            \item costs \$x00
            \item like an 8-yr old PC capability-wise
        \end{itemize}
\end{itemize}

Attacks:
\begin{itemize}
    \item MITM between TR device and user - need to access TR device with some
            other device which adversary can fake
    \item Hardware hacking attacks
        \begin{itemize}
            \item generate glitches by putting in strange environments
            \item device misbehaves
            \item hope misbehavior is useful
        \end{itemize}
    \item Kocher's timing attacker (1995)
        \begin{itemize}
            \item can extract RSA, DH keys from almost any sealed device
            \item basic idea: encryption time depends on key
        \end{itemize}
    \item Fault Analysis (1997 Boneh, DeMillo, Lipton)
        \begin{itemize}
            \item induce error in crypto operation
            \item ennorneous output leaks information about key
        \end{itemize}
    \item Differential Power Analysis
        \begin{itemize}
            \item idea: record power usage \underline{over time} in fine-grained
                    way
            \item very effective: identify path through code, easy to read out
                    variables
            \item devices without own power supply are particularly vulnerable
        \end{itemize}
\end{itemize}

\begin{subsektion}{Kocher's timing attacker (details)}
RSA is based on
\begin{verbatim}
Integer exp(Integer x, Integer d) {
    s = x
    r = 1
    for (k = 0; k < nbits; ++k) {
        b = d & (1 << k)
        if (b)      r *= s      <--- depends on bit value
        s = s*s
    }
    return r
}
\end{verbatim}
so play game to read off bits of \texttt{d}:\\
        guess, then if correct there will be a correlation to running time,
        otherwise will be wrong so still know bit (time depends on number of 1's
        in the key)

Can also use power consumption instead of running time for iteration of the loop
\end{subsektion}
\begin{subsektion}{Fault Analysis (details)}
To cause error, operate device outside spec
\begin{itemize}
    \item hot/cold
    \item overclock
    \item ``bad'' power supply
    \item hit with rubber mallet
\end{itemize}
Model:
\begin{itemize}
    \item attacker causes a single variable to change
    \item don't know which variable, when it changes, how it changes
\end{itemize}
Can attack RSA:

\sidenote{
{\bf RSA in practice}

Use Chinese Remainder Theorem for optimization: use the identity
    $$x^d \mod pq = a(x^d \mod p) + b(x^d \mod q)$$
    where $a \mod p = 1$, $a \mod q = 0$ and $b \mod p = 0$, $b \mod q = 1$
}
\begin{itemize}
    \item assume arbitrary error in computing $x^d \mod p$, getting
            $x^d \mod p + \Delta$, so result of full encrpytion is
            $$x^d \mod pq + a\Delta$$
    \item subtract off correct result to get $a\Delta$ (lots of ways to get
            correct result, since that's what happens when you don't mess with
            the device)
    \item now compute $\text{GCD}(a\Delta, pq)$. Note that $a$ is a multiple of
            $q$, so:
            \begin{align*}
            \text{GCD}(a\Delta, pq)&=\text{GCD}(q\alpha\Delta, pq)\\
              &=q\cdot\text{GCD}(\alpha\Delta,p)\\
              &=q\cdot1 = q
            \end{align*}
\end{itemize}
\end{subsektion}
\begin{subsektion}{Tamper-resistant software}
\begin{itemize}
    \item obfuscate to adversary
    \item difficult to do - lots more options for what to do than with hardware
\end{itemize}
\end{subsektion}
